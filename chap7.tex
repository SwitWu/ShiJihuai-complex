\chapter{共形映射\label{chap7}}
本章的主题是要对复平面上所有的单连通域在全纯等价的意义下进行分类.两个域$D_1$和$D_2$称为是\textbf{全纯等价}\index{Q!全纯等价}的,如果存在单叶的全纯函数$f:D_1\to\MC$,使得$f(D_1)=D_2$,即$f$一一地把$D_1$映为$D_2$.显然,这是一个等价关系. Riemann首先发现,所有的单连通域中只有两个等价类:一类仅由一个元素组成,那就是复平面$\MC$;另一类是除$\MC$以外的全部单连通域.换句话说,任何单连通域,只要不是整个复平面$\MC$,都是相互全纯等价的.这就是本章要证明的主要定理——Riemann映射定理. Riemann映射定理在复变函数论中具有十分重要的地位.在单连通域内研究保角变换下的某些不变量时,只需在最简单的单连通域——单位圆盘上研究就行了.另外,有些物理量的某些性质在保角变换下保持不变,应用Riemann映射定理讨论这类问题时,可把域尽量化简.

本章先介绍全纯函数的正规族理论,然后用它来证明Riemann映射定理,最后,作为Riemann映射定理的应用,我们给出把上半平面映射为多角形的Schwarz--Christoffel公式.

\section{正规族\label{sec7.1}}
我们从正规族的定义开始.
\begin{definition}\label{def7.1.1}
设$\MCF$是域$D$上的一个函数族,如果它的任意序列$\{f_n\}\subset \MCF$中一定包含一个在$D$上内闭一致收敛的子列$\{f_{n_k}\}$,就称$\MCF$是$D$上的一个\textbf{正规族}.\index{Z!正规族}
\end{definition}

为了给出正规族的特征,我们还需要下面的

\begin{definition}\label{def7.1.2}
设$\MCF$是域$D$上的一个函数族,如果存在常数$M$,使得对任意的$f\in\MCF$及$z\in D$,均有$|f(z)|\le M$,就说$\MCF$在$D$上是\textbf{一致有界}\index{Y!一致有界}的.如果对任意紧集$K\subset D$,存在与$K$有关的常数$M(K)$,使得对任意的$f\in\MCF$及$z\in K$,有$|f(z)|\le M(K)$,就说$\MCF$在$D$上是\textbf{内闭一致有界}\index{Y!一致有界!内闭一致有界}的.
\end{definition}

显然,在$D$上一致有界的函数族一定是内闭一致有界的,反之不然.
\begin{definition}\label{def7.1.3}
设$\MCF$是域$D$上的一个函数族,如果对任意$\varepsilon>0$,存在$\delta>0$,当$z_1,z_2\in D$,且$|z_1-z_2|<\delta$时,有
\[|f(z_1)-f(z_2)|<\varepsilon\]
对每个$f\in\MCF$都成立,就$\MCF$说在$D$上是\textbf{等度连续}\index{D!等度连续}的.
\end{definition}

关于一致有界且等度连续的函数列,有下面的\textbf{Arzela--Ascoli定理}\index{D!定理!Arzela--Ascoli定理}:
\begin{theorem}\label{thm7.1.4}
设$K$是$\MC$中的紧集,$\{f_n\}$是在$K$上一致有界且等度连续的函数列,那么$\{f_n\}$必有子列在$K$上一致收敛.
\end{theorem}
\begin{proof}
令$A\subset K$是可数集,且在$K$中稠密.记
\[A=\{\zeta_1,\zeta_2,\cdots\}.\]
由于$\{f_n(\zeta_1)\}$是有界数列,根据Bolzano--Weierstrass定理,从中可以取出收敛的子列,即有
\[f_{n_1}^{(1)}(z),f_{n_2}^{(1)}(z),\cdots\]
在$z=\zeta_1$处收敛;因为$\{f_{n_j}^{(1)}(\zeta_2)\}$有界,故有
\[f_{n_1}^{(2)}(z),f_{n_2}^{(2)}(z),\cdots\]
在$z=\zeta_2$处收敛.继续这样做下去,可得一串函数列:
\begin{align*}
&f_{n_1}^{(1)}(z),f_{n_2}^{(1)}(z),\cdots,f_{n_k}^{(1)}(z),\cdots;\\
&\cdots;\\
&f_{n_1}^{(s)}(z),f_{n_2}^{(s)}(z),\cdots,f_{n_k}^{(s)}(z),\cdots;\\
&\cdots.
\end{align*}
其中,后一序列是前一序列的子列,第$s$个序列在
\[\zeta_1,\zeta_2,\cdots,\zeta_s\]
处收敛.现在取对角线序列
\begin{equation}\label{eq7.1.1}
f_{n_1}^{(1)}(z),\cdots,f_{n_2}^{(2)}(z),\cdots,f_{n_k}^{(k)}(z),\cdots,
\end{equation}
它在$A$中的每一点处都收敛.

为符号简单起见,我们把函数列 \eqref{eq7.1.1} 记为$\{f_{n_k}\}$,它是$\{f_n\}$的一个子列,我们证明它在$K$上一致收敛.因为$\{f_n\}$在$K$上等度连续,故对任意$\varepsilon>0$,存在$\delta>0$,当$z_1,z_2\in K$,且$|z_1-z_2|<\delta$时,有$|f_n(z_1)-f_n(z_2)|<\varepsilon$对$n=1,2,\cdots$都成立.显然,$\bigcup_{z\in K}B\bigg(z,\frac\delta2\bigg)$是$K$的一个开覆盖,根据有限覆盖定理,可以选出有限个圆盘,设为
$B\bigg(z_j,\frac\delta2\bigg),j=1,\cdots,n$来覆盖$K$.今取$\zeta_j\in B\bigg(z_j,\frac\delta2\bigg),j=1,\cdots,n$,由于$\{f_{n_k}(\zeta_j)\},j=1,\cdots,n$收敛,由Cauchy收敛原理,存在$N$,当$l,k>N$时,有
\begin{equation}\label{eq7.1.2}
|f_{n_l}(\zeta_j)-f_{n_k}(\zeta_j)|<\varepsilon,j=1,\cdots,n.
\end{equation}
今任取$z\in K,z$必定属于$B\bigg(z_j,\frac\delta2\bigg),j=1,\cdots,n$中的一个,不妨设$z\in B\bigg(z_p,\frac\delta2\bigg)$. 因为$\zeta_p\in B\bigg(z_p,\frac\delta2\bigg)$,于是$|z-\zeta_p|<\delta$,因而
\begin{equation}\label{eq7.1.3}
|f_n(z)-f_n(\zeta_p)|<\varepsilon,n=1,2,\cdots.
\end{equation}
由 \eqref{eq7.1.2} 式和 \eqref{eq7.1.3} 式即知,当$l,k>N$时,便有
\begin{equation*}
|f_{n_l}(z)-f_{n_k}(z)|\le|f_{n_l}(z)-f_{n_l}(\zeta_p)|+
|f_{n_l}(\zeta_p)-f_{n_k}(\zeta_p)|+|f_{n_k}(\zeta_p)-f_{n_k}(z)|
<3\varepsilon.
\end{equation*}
这就证明了$\{f_{n_k}\}$在$K$上一致收敛.
\end{proof}

下面的Montel定理给出了全纯函数族是正规族的特征:
\begin{theorem}[(\textbf{Montel})]\label{thm7.1.5}\index{D!定理!Montel定理}
设$\MCF$是域$D$上的全纯函数族,那么$\MCF$是正规族的充分必要条件是$\MCF$在$D$上内闭一致有界.
\end{theorem}
\begin{proof}
先证必要性.如果$\MCF$是$D$上的正规族,但不是内闭一致有界的,那么存在一个紧集$K\subset D$,使得
\[\sup\{|f(z)|:z\in K,f\in\MCF\}=\infty.\]
因而存在序列$\{f_n\}\subset \MCF$,使得
\begin{equation}\label{eq7.1.4}
\sup\{|f_n(z)|:z\in K\}\ge n.
\end{equation}
由于$\MCF$是$D$上的正规族,故在$\{f_n\}$中存在子列$\{f_{n_k}\}$,它在$D$上内闭一致收敛,设其极限函数为$f$,则当$k>k_0$时,$|f_{n_k}(z)-f(z)|<1$在$K$上成立. $f$是$D$上的全纯函数,当然在$K$上有界,不妨设$|f(z)|\le M$($z\in K$).于是,当$z\in K$且$k>k_0$时,有
\begin{equation}\label{eq7.1.5}
|f_{n_k}(z)|\le |f_{n_k}(z)-f(z)|+|f(z)|<M+1.
\end{equation}
由 \eqref{eq7.1.4} 和 \eqref{eq7.1.5} 式就得到$n_k\le M+1$的矛盾.

现在证明充分性.任取$\{f_n\}\subset\MCF$,再取圆盘$\bar{B(z_0,r)}\subset D$,按假定,$\{f_n\}$在$\bar{B(z_0,r)}$上一致有界.利用引理 \ref{lemma4.1.8},即知$\{f_n'\}$在$\bar{B(z_0,r)}$上也一致有界,不妨设$|f_n'|\le M,z\in\bar{B(z_0,r)},n=1,2,\cdots$. 对任意$\varepsilon>0$,取$\delta=\frac\varepsilon M$,当$z_1,z_2\in B(z_0,r)$,且$|z_1-z_2|<\delta$时,有
\[|f_n(z_1)-f_n(z_2)|=\bigg|\int_{z_2}^{z_1}f_n'(z)\dz\bigg|\le M|z_1-z_2|<\varepsilon.\]
这就证明了$\{f_n\}$在$B(z_0,r)$上等度连续. 今取任意紧集$K\subset D$,由于$K$可用有限个上面这种圆盘来覆盖,因而$\{f_n\}$在$K$上也是等度连续的,当然$\{f_n\}$在$K$上也是一致有界的.由定理 \ref{thm7.1.4},$\{f_n\}$有子列$\{f_{n_k}\}$在$K$上一致收敛.为了取出在任意紧集上一致收敛的子列,取一列紧集$K_j$,使得
\[K_1\subset K_2\subset\cdots\subset K_j\subset\cdots\to D.\]
根据上面的讨论,先在$\{f_n\}$中取出在$K_1$上一致收敛的子列$\{f_{n_k}^{(1)}\}$,再从$\{f_{n_k}^{(1)}\}$中取出在$K_2$上一致收敛的子列$\{f_{n_k}^{(2)}\}$,继续这样做下去.最后,用对角线法即可取出在所有$K_j$上一致收敛的子列,这个子列便在任意紧集上一致收敛.
\end{proof}

\begin{xiti}
\item 设$\{f_n\}$是域$D$上的全纯函数列,并且在$D$上内闭一致有界.证明:若$\lim_{n\to\infty}f_n(z)$在$D$上处处存在,则$\{f_n\}$在$D$上内闭一致收敛.
\item 设$\{f_n\}$是域$D$上的全纯函数列,并且在$D$上内闭一致有界,$A=\{z=x+\ii y\in D:x,y\text{为有理数}\}$.证明:若$\lim_{n\to\infty}f_n(z)$在$A$上处处存在,则$\{f_n\}$在$D$上内闭一致收敛.
\item (\textbf{Vitali定理}\index{D!定理!Vitali定理})设$\{f_n\}$是域$D$上的全纯函数列,并且在$D$上内闭一致有界,$\{z_n\}$是$D$中彼此不同的点列,$\lim_{n\to\infty}z_n=z_0\in D$.证明:若$\lim_{n\to\infty}f_n(z)$在$\{z_n\}$上处处存在,则$\{f_n\}$在$D$上内闭一致收敛.
\item 设$\MCF$是域$D$上的全纯函数族,$z_0\in D$.证明:若
\begin{enuma}
  \item $\Re f(z)\ge0,\forall z\in D,f\in\MCF$;
  \item \label{xiti7.1.4.2}$f(z_0)=g(z_0),\forall f,g\in\MCF$,
\end{enuma}
则$\MCF$是$D$上的正规族.并举例说明条件 \ref{xiti7.1.4.2} 是不可去掉的.
\item 设$\MCF$是域$D$上的正规全纯函数族,$g$是整函数。证明:
$\{g\circ f:f\in\MCF\}$也是$D$上的正规族.
\item 设$D$是有界域,$0<M<\infty$. 证明:
\[\MCF=\bigg\{f\in H(D):\iint\limits_D|f(z)|^2\dx\dy\le M\bigg\}\]
是$D$上的正规族.
\item 设$\{f_n\}$是域$D$上的全纯函数列,并且在$D$上内闭一致有界.证明:若存在$z_0\in D$,使
得$\lim_{n\to\infty}f_n(z_0)=A$存在,并且$f_n(D)\subset\MC\backslash\{A\}$($n\in\MN$),则$\{f_n\}$在$D$上内闭一致收敛于常数$A$.
\end{xiti}

\section{Riemann映射定理\label{sec7.2}}
现在可以证明下面的Riemann映射定理:
\begin{theorem}[(\textbf{Riemann})]\label{thm7.2.1}\index{D!定理!Riemann映射定理}
设$G$是$\MC$中的单连通域,$G\ne\MC$.对于$G$中的任意点$a$,存在唯一的函数$f:G\to\MC$,使得
\begin{eenum}
  \item $f$在$G$中全纯且单叶;
  \item $f(a)=0,f'(a)>0$;
  \item $f(G)=B(0,1)$.
\end{eenum}
\end{theorem}
\begin{proof}
证明分下面四步进行:

(1) \hypertarget{thm7.2.1.1}{} 作一函数$\varphi:G\to\MC$,它在$G$中是全纯且单叶的,使得$\varphi(G)$是一有界域.事实上,因为$G\ne\MC$,故存在$b\notin G,b\ne\infty$. 在$G$中取
$\sqrt{z-b}$的单值全纯分支,记为$g(z)=\sqrt{z-b}$.显然,它在$G$中是单叶的.设$g(G)=E$,则由第 \ref{chap4} 章定理 \ref{thm4.4.6} 知,$E$是$\MC$中的域. $E$有一个简单的性质:如果$w\in E$,那么$-w\notin E$.因为如果$-w\in E$,则有$z_1,z_2\in G$,使得$\sqrt{z_1-b}=w,\sqrt{z_2-b}=-w$.两边平方后即得$z_1=z_2$,因而$w=-w$,即$w=0$.这是不可能的,因为$g$在$G$中没有零点.因为$a\in G$,所以$g(a)$是$E$的内点,故存在$\delta>0$,使得$B\big(g(a),\delta\big)\subset E$,因而$B\big(-g(a),\delta\big)\subset\MC\backslash E$.于是,当$z\in G$时,便有$|g(z)+g(a)|>\delta$.令
\[\varphi(z)=\frac1{g(z)+g(a)},\]
则当$z\in G$时,$|\varphi(z)|<\frac1\delta$,即$\varphi(G)\subset B\bigg(0,\frac1\delta\bigg)$,这说明$\varphi(G)$是一个有界域.

(2) \hypertarget{thm7.2.1.2}{} 由步骤 \hyperlink{thm7.2.1.1}{(1)} 所证,我们不妨假定$G$是有界域.在$G$上定义函数族
\[\MCF=\{f:f\mbox{在$G$上全纯且单叶},f(a)=0,f'(a)>0,f(G)\subset B(0,1)\}.\]
我们首先证明$\MCF$不是空集.事实上,因为$G$是有界域,故存在$R>0$,使得$G\subset B(0,R)$.若令
\[f(z)=\frac1{2R}(z-a),\]
由于$|f(z)|\le\frac1{2R}(|z|+|a|)<1$,故$f\in\MCF$,这证明$\MCF$非空.

现取$r>0$,使得$B(a,r)\subset G$.根据Cauchy不等式,对任意$f\in \MCF$,有
\[f'(a)<\frac1r\sup\{|f(z)|:z\in B(a,r)\}<\frac1r.\]
这说明数集$\{f'(a):f\in\MCF\}$有上界,设$M$为其上确界,则
\[M=\sup\{f'(a):f\in\MCF\}<\infty.\]
下面我们证明,存在$f_\ast\in \MCF$,使得$f_\ast'(a)=M$.事实上,存在$f_n\in\MCF$,使得$\lim_{n\to\infty}f_n'(a)=M $. 由于$\MCF$在$G$上一致有界,因而由定理 \ref{thm7.1.5},它是一个正规族,故在$\{f_n\}$中可以取出在$G$上内闭一致收敛的子列$\{f_{n_k}\}$,设其极限函数为$f_\ast$,即
\begin{equation}\label{eq7.2.1}
\lim_{k\to\infty}f_{n_k}(z)=f_\ast(z).
\end{equation}
由Weierstrass定理,$f_\ast\in H(G)$,且
\begin{equation}\label{eq7.2.2}
\lim_{k\to\infty} f_{n_k}'(z)=f_\ast'(z).
\end{equation}
由 \eqref{eq7.2.2} 式及$\lim_{k\to\infty}f_{n_k}'(a)=M$,即得
\begin{equation}\label{eq7.2.3}
f_\ast'(a)=M.
\end{equation}
现在证明$f_\ast\in\MCF$.因为每一个$f_{n_k}$都是$G$中的单叶全纯函数,由 \eqref{eq7.2.3} 式知$f_\ast'$不是常数,由定理 \ref{thm4.4.11} 知道$f_\ast$也是$G$中的单叶全纯函数.由 \eqref{eq7.2.1} 式易知$f_\ast(a)=0$,且$|f_\ast(z)|\le1,z\in G$.由最大模原理,$|f_\ast(z)|=1$不能成立,因而只有$|f_\ast(z)|<1$.这就证明了$f_\ast\in\MCF$.

(3) \hypertarget{thm7.2.1.3}{} 现在证明$f_\ast$即是定理中要找的函数,即要证明$f_\ast(G)=B(0,1)$.如果不是这样,设
\[f_\ast(G)=G_1\subset B(0,1),\]
那么$G_1$是$B(0,1)$中的一个单连通域,但$G_1\ne B(0,1)$.因而存在$u_0\in B(0,1),u_0\notin G_1$,由此即可引出一个矛盾.为此,令
\[v=\varphi_{u_0}=\frac{u_0-u}{1-\bar{u_0}u},\]
于是
\begin{align*}
&\varphi_{u_0}(u_0)=0,\\
&\varphi_{u_0}'(0)=-(1-|u_0|^2).
\end{align*}
$\varphi_{u_0}$把$G_1$映为$B(0,1)$中的单连通域$G_2$,把$u_0$映为原点,把原点映为$v_0=u_0$,所以$O\notin G_2$.再令
\[s=p(v)=\sqrt v,\]
于是
\begin{align*}
&s_0=p(v_0)=\sqrt{v_0},\\
&p'(v_0)=\frac1{2\sqrt{v_0}}.
\end{align*}
它在$G_2$中能分出单值全纯的分支,记$p(G_2)=G_3$,则$G_3$是$B(0,1)$中的单连通域,$O\notin G_3$.最后,令
\[w=q(s)=\frac{s_0}{|s_0|}\varphi_{s_0}(s),\]
于是
\[q'(s_0)=-\frac{s_0}{|s_0|}\frac1{1-|s_0|^2}.\]
它把$G_3$映为$B(0,1)$中的单连通域$G_4$,把$s_0$映为原点.于是,复合函数
\[w=(q\circ p\circ\varphi_{u_0}\circ f_\ast)(z)=w(z)\]
\begin{figure}[!ht]
\centering
\begin{tikzpicture}[thick,every node/.style={inner sep=2pt},scale=0.9,
>={Stealth[width=3pt]}]
\begin{scope}[xshift=-5.5cm]
\draw[rotate=55](0,0)ellipse(2.5 and 1.2);
\fill(60:1.6)circle(1pt)node[above right]{$a$};
\draw(-120:1.8)node{$G$}(0,-2.2)node{$z$};
\end{scope}
\draw[->](-3.9,0)--node[below]{$u=f_\ast (z)$}(-2.3,0);
\draw(0,0)circle(2);
\fill(0,0)node[below left]{$O$}circle(1pt)(1.2,0.1)circle(1pt)node[below]{$u_0$};
\draw[rotate=100](0,0)ellipse(1.2 and 0.7);
\draw(-0.1,-0.7)node{$G_1$};
\draw[->](2.3,0)--node[below]{$v=\varphi_{u_0(u)}$}(3.9,0);
\node at(0,-2.2){$u$};
\begin{scope}[xshift=6.1cm]
\draw(0,0)circle(2);
\fill(0,0)node[left]{$O$}circle(1pt)(95:1.2)node[right]{$v_0$}circle(1pt);
\draw(0.1,-1.1)node{$G_2$};
\draw(-0.1,-0.4)arc(100:260:0.65)arc(-100:90:1.65)arc(90:140:1.1)
arc(140:270:0.45)arc(90:20:0.6)arc(20:0:0.1)[bend left=40]to(-0.1,-0.4);
\node at(0,-2.2){$v$};
\end{scope}
\begin{scope}[yshift=-4.5cm]
\draw[->](-6,0)--node[below]{$s=p(v)$}(-4.6,0);
\draw[->](0.2,0)--node[below]{$w=q(s)$}(1.6,0);
\begin{scope}[xshift=-2.2cm]
\draw(0,0)node[left]{$O$}circle(2)(70:1)node[right]{$s_0$}(-0.1,-1.1)node{$G_3$};
\draw(-0.1,-0.6)[bend right=5]to(-0.8,-0.6)arc(100:180:0.15)
arc(180:230:0.9)arc(-130:10:1.35)arc(10:40:2.4)
arc(40:100:1)arc(100:240:0.5)arc(60:20:1)arc(20:-50:0.3)
[bend left=20]to(-0.1,-0.6);
\node at(0,-2.2){$s$};
\fill(0,0)circle(1pt)(70:1)circle(1pt);
\end{scope}
\begin{scope}[xshift=4cm]
\draw(0,0)node[below]{$O$}circle(2);
\draw[rotate=70](-0.2,-0.2)ellipse(1.3 and 1);
\node at(0,-2.2){$w$};
\fill(0,0)circle(1pt);\node at(-0.1,-1.2){$G_4$};
\end{scope}
\end{scope}
\end{tikzpicture}
\caption{\label{fig7.1}}
\end{figure}
把$G$映为$G_4$,且$w(a)=0$(见图 \ref{fig7.1}). $w(z)$显然是$G$上的单叶全纯函数.现在来计算$w'(a)$,根据复合函数的求导法则,有
\begin{align*}
w'(a)&=q'(s_0)p'(v_0)\varphi_{u_0}'(0)f_\ast'(a)\\
&=\frac{s_0}{|s_0|}\frac1{1-|s_0|^2}\frac1{2\sqrt{v_0}}(1-|u_0|^2)M\\
&=\frac{1-|u_0|^2}{2|s_0|(1-|s_0|^2)}M>0.
\end{align*}
所以$w\in\MCF$. 若记$u_0=\rho\ee^{\ii\theta}$,则$v_0=-u_0=\rho\ee^{\ii(\theta+\pi)},|s_0|=\big|\sqrt{v_0}\big|=\sqrt\rho
,1-|s_0|^2=1-\rho$,因而
\[w'(a)=\frac{1-\rho^2}{2\sqrt\rho(1-\rho)}M=\frac{1+\rho}{2\sqrt\rho}M>M.\]
这就是说,我们在$\MCF$中找到了一个函数$w$,它在$a$点的导数大于$M$,这是不可能的.这个矛盾说明假设不成立,于是$f_\ast(G)=B(0,1)$.
\end{proof}

(4) \hyperlink{thm7.2.1.4}{} 最后证明满足定理中三个条件的$f$是唯一的.如果还有函数$g$也满足这三个条件,令
\[h=f\circ g^{-1},\]
那么$h\big(B(0,1)\big)=B(0,1)$,因而$h\in\Aut\big(B(0,1)\big)$. 又因$h(0)=0$,因而由第 \ref{chap4} 章定理 \ref{thm4.5.5} 知道
\[h(z)=\ee^{\ii\theta}z.\]
但由于$h'(0)=\frac{f'(a)}{g'(a)}>0$,即知$h(z)=z$,从而
\begin{equation*}
f(z)\equiv g(z),z\in G.
\end{equation*}

由Riemann定理立刻可得
\begin{theorem}\label{thm7.2.2}
设$G$和$D$是$\MC$中的两个单连通域,如果它们都不是整个平面$\MC$,那么对于给定的$z_0\in G$和$w_0\in D$,存在唯一的函数$f$,它在$G$上单叶且全纯,$f(z_0)=w_0,f'(z_0)>0$,且$f(G)=D$.
\end{theorem}
\begin{proof}
由Riemann映射定理,存在唯一的全纯函数$g$,使得$\zeta=g(z)$把$G$一一地映为$|\zeta|<1$,且$g(z_0)=0,g'(z_0)>0$.同样道理,存在唯一的全纯函数$\zeta=h(w)$,它把$D$一一地映为$|\zeta|<1$,且$h(w_0)=0,h'(w_0)>0$.于是,函数$w=(h^{-1}\circ g)(z)=f(z)$即是符合定理条件的函数.
\end{proof}

这个定理断言,两个异于$\MC$的单连通域是全纯等价的.已经知道单连通域和多连通域一定不全纯等价,那么两个$n$($n>1$)连通域是不是全纯等价呢?一般来说,答案是否定的.例 \ref{exam6.1.6} 已为我们提供了这样的例子:即使两个最简单的二连通域——两个同心圆环$D=\{z\in\MC:r_1<|z|<r_2\}$和$G=\{w\in\MC:R_1<|w| <R_2\}$全纯等价也是有条件的,要求它们满足$\frac{r_2}{r_1}=\frac{R_2}{R_1}$.这从另一个角度说明了Riemann映射定理的重要性.

\begin{xiti}\hypertarget{xiti7.2}{}
\item (\textbf{推广的Liouville定理}\index{D!定理!推广的Liouville定理})设$D$是异于$\MC$的单连通域. 证明:若$f$是整函数,并且$f(\MC)\subset D$,则$f$是常值函数.
\item 设$D$是异于$\MC$的单连通域,$a\in D$.证明:若$f$将$D$双全纯地映为$B(0,1)$,并且$f(a)=0,f'(a)>0$,则
    \[\min_{z\in\partial D}|z-a|\le\frac1{f'(a)}\le\max_{z\in\partial D}|z-a|.\]
称$\frac1{f'(a)}$为$D$在$a$处的\textbf{映射半径}.\index{Y!映射半径}
\item 设$D$是异于$\MC$的单连通域,$a\in D$,$f$将$D$双全纯地映为$B(0,1)$,并且$f(a)=0,f'(a)>0$.证明:若$g$将$D$双全纯地映为$B(0,1),p=g^{-1}(0)$,则
    \[g(z)=\frac{g'(a)}{|g'(a)|}\frac{f(z)-f(p)}{1-\bar{f(p)}f(z)}.\]
\item 设$D$为异于$\MC$的凸域,$a\in D,\MCF=\{f\in H(D):f(a)=0,f'(a)>0\}$.证明:$\MCF$中满足$f(D)=B(0,1)$和$\Re f'(z)\ge0$($\forall z\in D$)的$f$最多只有一个.
\item 设$D$是异于$\MC$的单连通域,$a\in D$,$R$为$D$在$a$处的映射半径.证明:若$F\in H(D) $,$F(a)=0,F'(a)=1$,则$\sup_{z\in D}|F(z)|\ge R$.等号成立当且仅当$F$是将$D$映为$B(0,R)$的双全纯映射.
\item 证明:
\begin{enuma}
  \item 存在$B(0,1)$上的全纯函数$f$,使得$f\big(B(0,1)\big)=\MC$;
  \item 对于$a\in \MC$,存在$B(0,1)$上的全纯函数$f$,使得$f\big(B(0,1)\big)=\MC\backslash\{a\}$.
\end{enuma}
      并将这里的结论与习题 \hyperlink{xiti6.3}{6.3} 的第 \hyperlink{xiti6.3.3}{3} 题和第 \hyperlink{xiti6.3.3}{4} 题作比较.
\item 设$D$是异于$\MC$的单连通域,$a\in D$,$R$为$D$在$a$处的映射半径.证明:若$F\in H(D)$,$F(a)=0,F'(a)=1$,则
    \[\iint\limits_D|F'(z)|^2\dx\dy\ge\pi R^2.\]
等号成立当且仅当$F$是将$D$映为$B(0,R)$的双全纯映射.
\end{xiti}

\section{边界对应定理\label{sec7.3}}
Riemann映射定理断言,一定存在双全纯映射$f$把任一单连通域$G$(整个平面$\MC$除外)一一地映为单位圆盘$B(0,1)$.一个自然的问题是,$f$在把$G$映为$B(0,1)$的同时,是否也把$G$的边界$\partial G$映为单位圆盘的边界?一般来说,一个域的边界可以是相当复杂的,下面只就$G$的边界是一条简单闭曲线的情形来回答上面的问题.
\begin{theorem}[(\textbf{边界对应定理})]\label{thm7.3.1}\index{D!定理!边界对应定理}
设$G$是由一条简单闭曲线$\Gamma$所围成的域,如果$w=f(z)$把$G$双全纯地映为$B(0,1)$,那么$f$的定义可扩充到$\Gamma$上,使得$f\in C(\bar G)$,且把$\Gamma$一一地映为$|w|=1$,$\Gamma$关于$G$的正向对应于$f(\Gamma)$关于$B(0,1)$的正向.
\end{theorem}
\begin{proof}
证明分下面四步进行:

(1) \hypertarget{thm7.3.1.1}{} 先证明对于任意$\zeta\in\partial G,\lim_{\substack{z\to\zeta\\z\in G}}f(z)$存在.为此只要证明,如果$\lim_{z_n\to\zeta}f(z_n)=a,\lim_{z_n'\to\zeta}f(z_n')=b$,则必有$a=b$.

首先,容易看出,$a,b$都在单位圆周上.不然的话,若$a\in B(0,1)$,记$w_n=f(z_n)$,则$z_n=f^{-1}(w_n)$.由于$f^{-1}$在$B(0,1)$中连续,因而有
\[\zeta=\lim_{n\to\infty}z_n=\lim_{n\to\infty}f^{-1}(w_n)=\lim_{w_n\to a}f^{-1}(w_n)=f^{-1}(a),\]
这不可能.

现在设$a\ne b$,作分式线性变换$T$,使得$T\big(B(0,1)\big)=B(0,1),T(a)=\ee^{\ii\frac\pi4},T(b)=\ee^{\ii\frac54\pi}$.令$g(z)=T\big(f(z)\big)$,则$g$仍然把$G$双全纯地映为$B(0,1)$,但
\begin{align*}
&\lim_{z_n\to\zeta}g(z_n)=\lim_{z_n\to\zeta}T\big(f(z_n)\big)=T(a)=\ee^{\ii\frac\pi4},\\
&\lim_{z_n'\to\zeta}g(z_n')=\lim_{z_n'\to\zeta}T\big(f(z_n')\big)=T(b)=\ee^{\ii\frac54\pi}.
\end{align*}
\begin{figure}[!ht]
\centering
\subcaptionbox{\label{fig7.2a}}[0.48\textwidth]
{
\begin{tikzpicture}[thick,every node/.style={inner sep=2pt},scale=1.2,>={Stealth[width=3pt]}]
\begin{scope}[rotate=55]
\draw[->](0,2)arc(90:452:2 and 1.3);
\clip(0,2)arc(90:452:2 and 1.3);
\draw(-35:1)circle(0.8);
\end{scope}
\fill(110:1.6)node[right]{$z_0$}circle(1pt);
\node at(180:1.9){$G$};
\draw(24:0.95)node[right]{$\zeta$}--++(-0.73,0)node[left]{$\delta$};
\draw(49:1.1)node(a){$z_n'$}(12:0.45)node(b){$z_n$};
\draw[shorten >=1mm,shorten <=1.7mm,->](49:1.1)[bend right=10]to(12:0.45);
\node at(55:0.7){$l$};
\end{tikzpicture}
}
\subcaptionbox{\label{fig7.2b}}[0.48\textwidth]
{
\begin{tikzpicture}[thick,every node/.style={inner sep=2pt},scale=2,>={Stealth[width=3pt]}]
\draw[->](-1.1,0)--(0,0)node[above left]{$O$}--(1.3,0)node[below]{$x$};
\draw[->](0,-1.1)--(0,1.3)node[right]{$y$};
\draw(0,0)circle(1);
\draw(0,-0.3)node[below right]{$Q$}[bend right=20]to(0.3,0)node[below right]{$P$}
[bend right=20]to(0,0.3)node[above right]{$Q'$}[bend right=20]to(-0.3,0)node[above left]{$P'$}[bend right=20]to(0,-0.3);
\draw(0.7071,0.7071)circle(0.2)(-0.7071,-0.7071)circle(0.2);
\fill(0.7071,0.7071)circle(0.5pt)(-0.7071,-0.7071)circle(0.5pt)
(-0.63,-0.7)circle(0.5pt)(0.63,0.7)circle(0.5pt);
\draw(0.7071,0.7071)--++(80:0.2)node[above]{$\varepsilon$};
\draw(-0.7071,-0.7071)--++(80:0.2)node[above]{$\varepsilon$};
\draw(0,-0.3)[bend right=10]to(-0.63,-0.7);
\draw(0.3,0)[bend left=10]to(0.63,0.7)(-0.3,-0.6)node{$L$};
\draw[->,shorten <=3mm,shorten >=1mm](-1.3,-0.8)node{$\textrm e^{\textrm i\frac54\pi}$}--(-0.7071,-0.7071);
\draw[->,shorten <=2mm,shorten >=0.5mm](-0.7071,-1.1)
node{$g(z_n')$}--(-0.63,-0.7);
\draw[->,shorten <=3mm,shorten >=1mm](1.2,1)node{$\textrm e^{\textrm i\frac\pi4}$}--(0.7071,0.7071);
\draw[->,shorten <=3mm,shorten >=0.5mm](1.15,0.5)
node{$g(z_n)$}--(0.63,0.7);
\end{tikzpicture}
}
\caption{\label{fig7.2}}
\end{figure}
对任给的$\varepsilon>0$,必有$\delta>0$,当$0<|z_n-\zeta|<\delta,0<|z_n'-\zeta|<\delta $时,$\big|g(z_n)-\ee^{\ii\frac\pi4}\big|<\varepsilon,\big|g(z_n')-\ee^{\ii\frac54\pi}\big|
<\varepsilon$.现设$z_0=g^{-1}(0)$,则$z_0\in G$.取充分小的$\delta>0$,使得$G_\delta=B(\zeta,\delta)\cap G$不包含$z_0$.取$z_n,z_n'\in G_\delta$,则$g(z_n),g(z_n')$分别在以$\ee^{\ii\frac\pi4}$和$\ee^{\ii\frac54\pi}$为中心、以$\varepsilon$为半径的圆盘中.用位于$G_\delta$中的连续曲线$l$连接$z_n$和$z_n'$,则它的像$L=g(l)$是连接$g(z_n)$和$g(z_n')$的$B(0,1)$中的连续曲线,它不会经过原点,因此必和实轴、虚轴相交,设交点分别为$P$和$Q$(见图 \ref{fig7.2}).用$\wideparen{PQ'},\wideparen{P'Q'},\wideparen{P'Q}$分别记$\wideparen{PQ}$相对于实轴、原点和虚轴对称的弧段,由这四段弧围成的域记为$D$,显然$O\in D$.

令
\[F(w)=\big(g^{-1}(w)-\zeta\big)\big(\bar{g^{-1}(\bar w)}-\bar \zeta\big)
\cdot\big(g^{-1}(-w)-\zeta\big)\big(\bar{g^{-1}(-\bar w)}-\bar \zeta\big),\]
则$F\in H\big(B(0,1)\big)$. 我们来估计$F$在$\partial D$上的值,任取$w\in\wideparen{PQ}$,则$g^{-1}(w)\in l$,因此$|g^{-1}(w)-\zeta|<\delta$. 如果记$M=\sup_{z\in\partial G}|z-\zeta|$,那么因为$g^{-1}(\bar w),g^{-1}(-w),g^{-1}(-\bar w)$都属于$G$,因而有
\begin{align*}
&|g^{-1}(\bar w)-\zeta|\le M,\\
&|g^{-1}(-w)-\zeta|\le M,\\
&|g^{-1}(-\bar w)-\zeta|\le M,
\end{align*}
于是$|F(w)|\le \delta M^3$.同样道理,当$w$属于其他三段弧时也有同样的估计,即在$\partial D$上有
\[|F(w)|\le\delta M^3.\]
因而由最大模原理,上式在$D$中也成立.特别有$|F(0)|\le\delta M^3$,即$|g^{-1}(0)-\zeta|^4\le\delta M^3$.让$\delta\to0$,即得$g^{-1}(0)=\zeta$,这不可能.因而$a=b$,于是$\lim_{\substack{z\to\zeta\\z\in G}}f(z)$存在.

\noindent\begin{minipage}{0.7\textwidth}\parindent=2em
(2) \hypertarget{thm7.3.1.2}{}  现在对任意$\zeta\in\partial G$,定义
\[f(\zeta)=\lim_{\substack{z\to\zeta\\z\in G}}f(z).\]
这样,$f$在$\bar G$上都有了定义,我们证明$f\in C(\bar G)$.为此,只需证明对任意$\zeta\in\partial G,\lim_{\substack{\xi\to\zeta\\\xi\in \partial G}}f(\xi)=f(\zeta)$就行了.如图
\ref{fig7.3} 所示,固定$\zeta\in\partial G$,对任意$\varepsilon>0$,存在$\delta>0$,当$z\in B(\zeta,\delta)\cap G$时,$|f(z)-f(\zeta)|<\varepsilon$.今取$\xi\in B(\zeta,\delta)\cap\partial G$,对于这个$\xi$,有相应的$\delta_1$,当$z\in B(\xi,\delta_1)\cap G$时,有$|f(\xi)-f(z)|<\varepsilon$.今取$z\in B(\zeta,\delta)\cap B(\xi,\delta_1)\cap G$,便有
\end{minipage}
\begin{minipage}{0.3\textwidth}
\centering
\begin{tikzpicture}[thick,every node/.style={inner sep=2pt},scale=1.3,>={Stealth[width=3pt]}]
\begin{scope}[rotate=60]
\draw(0,0)ellipse(2 and 1.1);
\draw({2*cos(-59)},{1.1*sin(-59)})circle(0.5);
\draw({2*cos(-49)},{1.1*sin(-49)})circle(0.6);
\fill({2*cos(-59)},{1.1*sin(-59)})coordinate(A) circle(0.7pt)node[right]{$\xi$};
\fill({2*cos(-49)},{1.1*sin(-49)})coordinate(B) circle(0.7pt)node[right]{$\zeta$};
\fill({1.6*cos(-53)},{0.7*sin(-53)})circle(0.7pt)node[right]{$z$};
\end{scope}
\draw(-120:1)node[below]{$G$};
\draw(B)--++(150:0.6)node[left]{$\delta$}(A)--++(-70:0.5)node[below]{$\delta_1$};
\node at(1.1,-0.8){$\partial G$};
\end{tikzpicture}
\captionof{figure}{\label{fig7.3}}
\end{minipage}
\[|f(\zeta)-f(\xi)|\le|f(\zeta)-f(z)|+|f(z)-f(\xi)|<2\varepsilon,\]
这就证明了$f\in C(\bar G)$.

(3) \hypertarget{thm7.3.1.3}{} 前面已经证明,对每个$\zeta\in\partial G,|f(\zeta)|=1$,这说明$f$把$\partial G$映入$\partial B(0,1)$.现在证明,如果$\zeta,\zeta'\in\partial G,\zeta\ne\zeta'$,那么$f(\zeta)\ne f(\zeta')$.取充分小的$\varepsilon>0$,使得$B(\zeta,\varepsilon)\cap B(\zeta',\varepsilon)=\varnothing$,由于$f$是单叶的,因而
\[f\big(B(\zeta,\varepsilon)\cap G\big)\cap f\big(B(\zeta',\varepsilon)\cap G\big)=
\varnothing,\]
这就证明了$f(\zeta)\ne f(\zeta')$.

由于$f$是$\bar G$上的一一连续映射,所以$f(\bar G)$是紧集.注意到$B(0,1)\subset f(\bar G)\subset\bar{B(0,1)}$,即知$f(\bar G)=\bar{B(0,1)}$.由习题 \hyperlink{xiti1.7}{1.7} 的第 \hyperlink{xiti1.7.7}{7} 题便知$f^{-1}$也是$\bar{B(0,1)}$上的一一连续映射.

(4) \hypertarget{thm7.3.1.4}{} 最后证明$f$保持边界的方向不变.在$\partial G$上沿着关于$G$的正方向取三点$z_1,z_2,z_3$,它们在$f$下的像分别为$w_1,w_2,w_3$,半径$Ow_1,Ow_2,Ow_3$的原像分别是曲线弧段$\wideparen{z_0z_1}$,$\wideparen{z_0z_2}$,$\wideparen{z_0z_3}$(图 \ref{fig7.4}).由于$f'(z_0)\ne0$,$f$在$z_0$处具有保角性,而且保持角度的方向不变,因此$w_1,w_2,w_3$也是沿着关于$B(0,1)$的正方向.
\end{proof}
\begin{figure}[!ht]
\centering
\subcaptionbox{\label{fig7.4a}}[0.48\textwidth]
{
\begin{tikzpicture}[thick,every node/.style={inner sep=2pt},rotate=60,scale=1.3,>={Stealth[width=3pt]}]
\draw(0,0)ellipse(2 and 1);
\coordinate(z0)at(0.2,0.2);
\coordinate(z3)at({2*cos(-60)},{sin(-60)});
\coordinate(z2)at({2*cos(-100)},{sin(-100)});
\coordinate(z1)at({2*cos(-150)},{sin(-150)});
\draw(z0)node[above left]{$z_0$}arc(90:0:0.3)[bend right=30]to(z3)node[right]{$z_3$};
\draw(z0)[bend left=20]to(-0.2,-0.2)arc(120:180:0.2)[bend left=20]to(z2)
node[below right]{$z_2$};
\draw(z0)[bend right=5]to(-0.2,0.1)arc(110:140:0.4)arc(-40:-100:1.3)
[bend right=20]to(z1)node[below right]{$z_3$};
\node at(-1.6,0.2){$G$};
\end{tikzpicture}
}
\subcaptionbox{\label{fig7.4b}}[0.48\textwidth]
{
\begin{tikzpicture}[thick,every node/.style={inner sep=2pt},scale=1.3,>={Stealth[width=3pt]}]
\draw(0,0)coordinate(O)node[left]{$O$}circle(2);
\draw(O)--++(-45:2)node[below right]{$w_1$}
(O)--++(-5:2)node[right]{$w_2$}(O)--++(60:2)node[above right]{$w_3$};
\end{tikzpicture}
}
\caption{\label{fig7.4}}
\end{figure}

后面我们将应用边界对应定理来推导把上半平面映为多角形域的Schwarz--Christoffel公式.

在实际应用中,下面的边界对应定理的逆定理很有用.
\begin{theorem}\label{thm7.3.2}
设$G$和$D$分别是由可求长简单闭曲线$\gamma$和$\Gamma$围成的域,如果$f\in H(G)\cap C(\bar G)$,且把$\gamma$一一地映为$\Gamma$,那么$w=f(z)$把$G$一一地映为$D$,并且使$\gamma$关于$G$的正向对应于$\Gamma$关于$D$的正向.
\end{theorem}
\begin{proof}
任取$w_0\in D$,先证明$f(z)-w_0$在$G$中有且只有一个零点.由辐角原理,$f(z)-w_0$在$G$中的零点个数为
\[N=\frac1{2\pi}\Delta_\gamma\Arg\big(f(z)-w_0\big)=\frac1{2\pi}\Delta_\Gamma
\Arg(w-w_0)=\pm1,\]
右端等于$\pm1$是因为$w_0\in D$之故.但因$N$是零点的个数,不能取$-1$,故有$N=1$.这就证明了$f(z)-w_0$在$G$内有且只有一个零点,且当$z$沿$\gamma$的正向转一圈时,$w$也沿$\Gamma$的正向转一圈.容易看出,当$w_0\notin\bar D$时,$N=0$,即$f$不会把$G$中的点映到$D$的外部去.
当$w_0\in\Gamma$时,如果存在$z_0\in G$,使得$f(z_0)=w_0$,那么$w_0$必须是$f(G)$的内点,这不可能.这就证明了定理的断言.
\end{proof}

下面看一个简单的例子:
\begin{example}\label{exam7.3.3}
问$w=z^2$把由圆周$\bigg|z-\frac12\bigg|=\frac12$围成的域变成什么样的域?
\end{example}
\begin{figure}[!ht]
\centering
\subcaptionbox{\label{fig7.5a}}[0.48\textwidth]
{
\begin{tikzpicture}[thick,every node/.style={inner sep=2pt},scale=4,>={Stealth[width=3pt]}]
\draw[->](0,0)node[left]{$O$}--(0.5,0)node[below]{$\frac12$}--(1,0)node[below right]{$1$}--(1.3,0)node[below]{$x$};
\draw(0.5,0)circle(0.5);\fill(0.5,0)circle(0.25pt);
\draw(0,0)--node[above]{$r$}(30:{cos(30)});
\draw(0.15,0)arc(0:30:0.15);\node at(15:0.2){$\theta$};
\end{tikzpicture}
}
\subcaptionbox{\label{fig7.5b}}[0.48\textwidth]
{
\begin{tikzpicture}[thick,every node/.style={inner sep=2pt},xscale=4,yscale=3.2,>={Stealth[width=3pt]}]
\draw[->](-0.2,0)--(0,0)node[below right]{$O$}--(1,0)node[below right]{$1$}--(1.3,0)node[below]{$x$};
\draw[->](0,-0.7)--(0,0.5)node[above left]{$\frac12$}--(0,0.8)node[right]{$y$};
\draw[domain=0:2*pi,samples=500]plot({(1+cos(\x r))*cos(\x r)/2},{(1+cos(\x r))*sin(\x r)/2});
\end{tikzpicture}
}
\caption{\label{fig7.5}}
\end{figure}
\begin{solution}
从图 \ref{fig7.5}\subref{fig7.5a} 可以看出,圆周$\bigg|z-\frac12\bigg|=\frac12$的极坐标方程为$r=\cos\theta\bigg(-\frac\pi2\le\theta\le\frac\pi2\bigg)$,这里,$z=r\ee^{\ii\theta}$. 若记$w=\rho\ee^{\ii\varphi}$,则$w=z^2=r^2\ee^{2\ii\theta}$,所以$\rho=r^2,\varphi=2\theta$,故$r=\cos\theta$经变换后的极坐标方程为
\[\rho=\frac12(1+\cos\varphi),\]
这是一条心脏线(图 \ref{fig7.5}\subref{fig7.5b}). 由定理 \ref{thm7.3.2},$w=z^2$把圆周$\bigg|z-\frac12\bigg|=\frac12$的内部一一地映成上述心脏线的内部.
\end{solution}

\begin{xiti}\hypertarget{xiti7.3}{}
\item 利用Schwarz对称原理和边界对应定理证明:将$B(0,1)$映为自身的双全纯映射一定是分式线性变换.
\item \hypertarget{xiti7.3.2}{} 证明:若$f$将圆环$\{z\in\MC:r_1<|z|<r_2\}$双全纯地映为圆环
$\{z\in\MC:R_1<|z|<R_2\}$,则$f$将闭圆环$\{z\in\MC:r_1\le|z|\le r_2\}$同胚地映为闭圆环$\{z\in\MC:R_1\le|z|\le R_2\}$.
\item 设$D$是由简单闭曲线所围成的单连通域,$z_1,z_2,z_3\in\partial D$是彼此不同的三点,按$\partial D$的正向排列。证明:若$w_1,w_2,w_3\in\partial B(0,1)$是彼此不同的三点,按$\partial B(0,1)$的正向排列,则存在唯一的$\varphi$,将$D$双全纯地映为$B(0,1)$,将$\bar D$同胚地映为$\bar{B(0,1)}$,并且$f(z_k)=w_k,k=1,2,3$.
\item (边界对应原理)设$D$是由简单闭曲线$\gamma$所围成的单连通域($\gamma$不必可求长),$f\in H(D)\cap C(\bar D)$.证明:若$f$将$\gamma$一一地映为简单闭曲线$\Gamma$,则$f$将$D$双全纯地映为由$\Gamma$所围成的单连通域$G$.将这里的结论与习题 \hyperlink{xiti4.4}{4.4} 中的第
     \hyperlink{xiti4.4.17}{17} 题作比较.
\item 设$f\in H\big(B(0,1)\big),f(0)=0,f'(0)=a>0$. 证明:若$f\big(B(0,1)\big)\subset B(0,1)$,则$f$在$B\bigg(0,\frac a{1+\sqrt{1-a^2}}\bigg)$上双全纯.
\end{xiti}

\section{Schwarz--Christoffel公式\label{sec7.4}}
Riemann映射定理断言,任意两个异于$\MC$的单连通域都可通过双全纯映射把一个变成另一个,但是具体写出这个映射却不是一件容易的事.本节介绍的Schwarz--Christoffel公式给出了把上半平面映为多角形域的变换.

设$G$是$w$平面上以$w_1,\cdots,w_n$为顶点的多角形域,$w=f(z)$是把$z$平面的上半平面$D$一一地映为$G$的双全纯映射.为了求得$f$的具体表达式,我们必须了解$f$的一些简单性质.
\begin{lemma}\label{lemma7.4.1}
  存在把上半平面$D$一一地映为多角形域$G$的双全纯映射$w=f(z)$,它在$\bar D$上连续,且把实轴一一地映为$ \partial G$.
\end{lemma}
\begin{proof}
任取$a\in D$,则分式线性变换
\[\zeta=\frac{z-a}{z-\bar a}\]
把$D$一一地映为$|\zeta|<1$,它在$\bar D$上连续,且把实轴一一地映为$|\zeta|=1$.根据Riemann映射定理和边界对应原理,存在双全纯函数$w=g(\zeta)$,它把$|\zeta|<1$一一地映为$G$,且$g$在$|\zeta|\le1$上连续,并把$|\zeta|=1$一一地映为$\partial G$.于是,函数
\[w=g\bigg(\frac{z-a}{z-\bar a}\bigg)=f(z)\]
即符合引理的要求.
\end{proof}

设$f$是引理 \ref{lemma7.4.1} 中的函数,那么在$z$平面的实轴上必有$n$个点$a_1,\cdots,a_n$,使得
\[f(a_1)=w_1,f(a_2)=w_2,\cdots,f(a_n)=w_n.\]
我们先弄清$f$在$a_k$点邻域中的性质.

\begin{lemma}\label{lemma7.4.2}
设$\Omega$是具有角点$w_0$的域,在$w_0$的邻域中,$\Omega$的边界由两个直线段构成,它们的交角为$\alpha\pi$.设双全纯映射$w=f(z)$把上半平面映为$\Omega$,把实轴上的点$z_0$映为$w_0$,那么在$z_0$的邻域内,$f$可表示为
\begin{equation}\label{eq7.4.1}
f(z)=w_0+(z-z_0)^\alpha\{c_0+c_1(z-z_0)+\cdots\},c_0\ne0.
\end{equation}
\end{lemma}
\begin{proof}
令$\eta=(w-w_0)^{\frac1\alpha}$,它可以在$\Omega$中分出单值的全纯分支,它把$w_0$映为原点,且把$\Omega$的两条边界线所张的角扩大成$\pi$.因此
\begin{equation}\label{eq7.4.2}
\eta=\eta(z)=\big(f(z)-w_0\big)^{\frac1\alpha}
\end{equation}
把$z_0$的邻域在上半平面中的那部分变到$\eta$平面上$\eta=0$的一个邻域的某个半平面,对应于$z$平面中实轴上的线段的是一段直线段.由引理 \ref{lemma7.4.1} 知,它连续到实轴上,故由Schwarz对称原理,$\eta=\eta(z)$可以全纯开拓到$z_0$的整个邻域中,因而在$z_0$的邻域中有展开式
\begin{equation}\label{eq7.4.3}
\eta(z)=b_1(z-z_0)+b_2(z-z_0)^2+\cdots,b_1\ne0,
\end{equation}
这里,没有常数项是因为$\eta(z_0)=0,b_1\ne0$是因为$b_1=\eta'(z_0)\ne0$. 由 \eqref{eq7.4.2} 式和 \eqref{eq7.4.3} 式即得
\begin{equation}\label{eq7.4.4}
\begin{aligned}
f(z)&=w_0+\big(\eta(z)\big)^\alpha\\
&=w_0+(z-z_0)^\alpha\{b_1+b_2(z-z_0)+\cdots\}^\alpha.
\end{aligned}
\end{equation}
如果记$g(z)=b_1+b_2(z-z_0)+\cdots$,由于$g(z_0)=b_1\ne0$,故在$z_0$的邻域中$\big(g(z)\big)^\alpha$能分出单值的全纯分支,把$\big(g(z)\big)^\alpha$在$z_0$的邻域中展开成幂级数
\[\big(g(z)\big)^\alpha=c_0+c_1(z-z_0)+\cdots.\]
再由 \eqref{eq7.4.4} 式即得 \eqref{eq7.4.1} 式.
\end{proof}

现在可以证明本节的主要定理:
\begin{theorem}[(\textbf{Schwarz--Christoffel})]\label{thm7.4.3}\index{D!定理!Schwarz--Christoffel定理}
设双全纯函数$w=f(z)$把上半平面$D$一一地映为多角形域$G$,且$f$在$\bar D$上连续.如果$G$在其顶点$w_k$($k=1,\cdots,n$)处的顶角是$\alpha_k\pi$($0<\alpha_k<2$),实轴上与$w_k$对应的点是$a_k$($k=1,\cdots,n$),那么$f$可表示为
\begin{equation}\label{eq7.4.5}
f(z)=C\int_{z_0}^z(z-a_1)^{\alpha_1-1}\cdots(z-a_n)^{\alpha_n-1}\dz+C_1,
\end{equation}
其中,$z_0,C,C_1$是三个常数.
\end{theorem}
\begin{proof}
由于$f$把实轴一一地映为$G$的边界,所以当$z$在$(a_k,a_{k+1})$中变动时,$f(z)$在线段$w_kw_{k+1}$上变动,即$f$把直线段映为直线段.根据Schwarz对称原理,$f$可以跨过$(a_k,a_{k+1})$,全纯开拓到下半平面$D'$中去.即可以得一函数$f_k$,它满足条件:
\begin{eenum}
  \item $f_k$在$D\cup D'\cup(a_k,a_{k+1})$中全纯;
  \item 当$z\in D$时,$f_k(z)=f(z),f_k$把下半平面$D'$映成$G$关于线段$w_kw_{k+1}$对称的多角形域$G'$;
  \item 当$z\in D\cup D'\cup(a_k,a_{k+1})$时,$f_k'(z)\ne0$.
\end{eenum}
让$k=1,\cdots,n$,我们就得到$n$个函数$f_1,\cdots,f_n$,它们在$D$内都等于$f$,而在$D'$中是不相等的.

我们来研究$f_k$和$f_{k+1}$在$D'$中的关系. 任取$z_0\in D'$,则$\bar{z_0}\in D$,于是$f_k(z_0)$是$f(\bar{z_0})$关于线段$w_kw_{k+1}$的对称点,$f_{k+1}(z_0)$是$f(\bar{z_0})$关于线段$w_{k+1}w_{k+2}$的对称点(图 \ref{fig7.6}).因此,$f_{k}(z_0)$可以看成是由$f_{k+1}(z_0)$经过两条直线的对称变换得到的.显然,有
\begin{align*}
f_k(z_0)-w_{k+1}&=\ee^{2\ii\theta}\big(f(\bar{z_0})-w_{k+1}\big)\\
&=\ee^{2\ii(\theta+\varphi)}\big(f_{k+1}(z_0)-w_{k+1}\big)\\
&=\ee^{2\ii\alpha_{k+1}\pi}\big(f_{k+1}(z_0)-w_{k+1}\big).
\end{align*}
\begin{figure}[!ht]
\centering
\subcaptionbox{\label{fig7.6a}}[0.48\textwidth]
{
\begin{tikzpicture}[thick,every node/.style={inner sep=2pt},scale=1.4,
>={Stealth[width=3pt]}]
\draw(-0.5,0)--(0,0)node[below]{$a_k$}--(1,0)node[below]{$a_{k+1}$}
--(2,0)node[below]{$a_{k+2}$}--(3,0);
\fill(0,0)circle(0.7pt)(1,0)circle(0.7pt)(2,0)circle(0.7pt)
(1.2,1)circle(0.7pt)node[right]{$\bar{z_0}$}
(1.2,-1)circle(0.7pt)node[right]{$z_0$};
\end{tikzpicture}
}
\subcaptionbox{\label{fig7.6b}}[0.48\textwidth]
{
\begin{tikzpicture}[thick,every node/.style={inner sep=2pt},scale=1.4,
>={Stealth[width=3pt]}]
\coordinate(O)at(0,0);\coordinate(A)at(30:1.2);
\coordinate(B)at(70:2);\coordinate(C)at(100:2.3);
\coordinate(D)at(110:1.5);\coordinate(E)at(150:2);
\coordinate(F)at(190:1.5);\coordinate(G)at(200:2.3);
\coordinate(H)at(230:1.9);\coordinate(I)at(265:1.2);
\coordinate(J)at(280:2);\coordinate(K)at(320:1.5);
\coordinate(L)at(330:2.2);\coordinate(M)at(355:2);
\draw(O)--(A)node[above right]{$w_{k+2}$}--(B)--(C)--(D)
--(E)node[left]{$w_k$}--(F)--(G)--(H)--(I)--(O)(O)--(J)--(K)
--(L)--(M)--(A)(O)--(E);
\coordinate(a)at(85:1.3);\coordinate(b)at(-10:1.4);
\coordinate(c)at(-150:1.5);
\draw(c)--(a)--(b);
\draw[densely dashed](O)--(a)node[above]{$f(\bar{z_0})$}(O)--(b)node[below]{$f_{k+1}(z_0)$}(O)--(c)
node[below]{$f_k(z_0)$};
\node[fill=white,right]at(0.05,-0.12){$w_{k+1}$};
\draw(30:0.25)arc(30:85:0.25)(85:0.2)arc(85:150:0.2);
\draw(53:0.4)node{$\varphi$}(115:0.35)node{$\theta$};
\end{tikzpicture}
}
\caption{\label{fig7.6}}
\end{figure}
由于$z_0$是$D'$中的任意点,故可设
\[f_k(z)=\alpha_kf_{k+1}(z)+\beta_k,z\in D',\]
其中,$\alpha_k=\ee^{2\ii\alpha_{k+1}\pi},\beta_k=(1-\ee^{2\ii\alpha_{k+1}\pi})w_{k+1}$,所以
\begin{align*}
  &f_k'(z)=\alpha_kf_{k+1}'(z),\\
  &f_k''(z)=\alpha f_{k+1}''(z),
\end{align*}
因而
\[\frac{f_k''(z)}{f_k'(z)}=\frac{f_{k+1}''(z)}{f_{k+1}'(z)},z\in D'.\]
这说明$f_k$($k=1,\cdots,n$)在$D'$中虽然都不相等,但$\frac{f_k''}{f_k'}$($k=1,\cdots,n$)在$D'$中却都是一样的,它们在$D$中当然都等于$\frac{f''}{f'}$. 这样,我们就得到了一个单值函数$g$,它在$D'$中等于$\frac{f_k''}{f_k'}$,在$D$中
\[g(z)=\frac{f_k''(z)}{f_k'(z)}.\]
故$g$在$\MC$中除去$a_1,\cdots,a_n$外是全纯的.

下面我们证明,$a_k$($k=1,\cdots,n$)是$g$的$1$阶极点.事实上,由引理 \ref{lemma7.4.2},$f$在$a_k$的邻域中可展开为
\[f(z)=w_k+(z-a_k)^{\alpha_k}h(z),\;h(a_k)\ne0,\]
这里,$h$是$a_k$的邻域中的全纯函数.于是
\begin{align*}
f'(z)&=\alpha_k(z-a_k)^{\alpha_k-1}h(z)+(z-a_k)^{\alpha_k}h'(z)\\
&=(z-a_k)^{\alpha_k-1}[\alpha_kh(z)+(z-a_k)h'(z)]\\
&=(z-a_k)^{\alpha_k-1}q(z),
\end{align*}
这里,$q(z)=\alpha_kh(z)+(z-a_k)h'(z)$是$\alpha_k$的邻域中的全纯函数,而且$q(a_k)=\alpha_kh(a_k)\ne0$.因而
\[f''(z)=(\alpha_k-1)(z-a_k)^{\alpha_k-2}q(z)+(z-a_k)^{\alpha_k-1}q'(z).\]
于是,在$a_k$的邻域中有
\begin{equation}\label{eq7.4.6}
g(z)=\frac{f''(z)}{f'(z)}=\frac{\alpha_k-1}{z-a_k}+\frac{q'(z)}{q(z)}.
\end{equation}
因为$\frac{q'(z)}{q(z)}$在$a_k$的邻域中全纯,可以展开成幂级数,所以 \eqref{eq7.4.6} 式就
是$g$在$a_k$处的Laurent展开式,它说明$a_k$是$g$的$1$阶极点,$g$在$a_k$处的残数为$\alpha_k-1$.

再看$g$在$z=\infty$处的情况.由于$f$把$\infty$映入多角形的边界,因此$f$在$\infty$处全纯,它在$\infty$的邻域中有展开式
\begin{align*}
&f(z)=b_0+\frac{b_m}{z^m}+\frac{b_{m+1}}{z^{m+1}}+\cdots,b_m\ne0,\\
&f'(z)=-\frac{mb_m}{z^{m+1}}-\frac{(m+1)b_{m+1}}{z^{m+2}}-\cdots=-\frac1{z^{m+1}}p(z),
\end{align*}
这里,$p$在$z=\infty$的邻域内全纯,且$p(\infty)\ne0$. 由此可得
\[f''(z)=\frac{m+1}{z^{m+2}}p(z)-\frac1{z^{m+1}}p'(z),\]
因而
\[g(z)=\frac{f''(z)}{f(z)}=-\frac{m+1}z+\frac{p'(z)}{p(z)}.\]
由此即得
\begin{equation}\label{eq7.4.7}
\lim_{z\to\infty}g(z)=0.
\end{equation}

现在令
\[G(z)=g(z)-\sum_{k=1}^n\frac{\alpha_k-1}{z-a_k},\]
它是一个整函数,由 \eqref{eq7.4.7} 式即知$G(z)\equiv0$,即
\[g(z)=\sum_{k=1}^n\frac{\alpha_k-1}{z-a_k}\]
由于在上半平面中,$g(z)=\frac{f''(z)}{f'(z)}$,因而有
\[\frac{f''(z)}{f'(z)}=\sum_{k=1}^n\frac{\alpha_k-1}{z-a_k},z\in D.\]
让上式两边沿上半平面内任一条从$z_0$到$z$的可求长曲线积分,可得
\begin{align*}
\log f'(z)&=\sum_{k=1}^n(\alpha_k-1)\log(z-a_k)+C'\\
&=\log[C(z-a_1)^{\alpha_1-1}\cdots(z-a_n)^{\alpha_n-1}],
\end{align*}
即
\[f'(z)=C(z-a_1)^{\alpha_1-1}\cdots(z-a_n)^{\alpha_n-1},\]
由此即得
\[f(z)=C\int_{z_0}^zC(z-a_1)^{\alpha_1-1}\cdots(z-a_n)^{\alpha_n-1}\dz+C_1.\]
这就是所要证明的.
\end{proof}

在Schwarz--Christoffel公式中,用以表达$f$的是$z$平面实轴上的点$a_1,\cdots,a_n$.但在具体问题中,通常给出的是多角形的顶点$w_k$($k=1,\cdots,n$),而$a_k$却是未知的,如何确定$a_k$是解决问题的关键.在实际解题时,我们往往在实轴上任取三点$a_1<a_2<a_3$,使其与多角形的三个顶点$w_1,w_2,w_3$相对应,而其他的$a_4,\cdots,a_n$及$C,C_1$则由具体问题中的条件来确定.之所以能这样做,其依据是下面的定理:

\begin{theorem}\label{thm7.4.4}
 对于实轴上任意三点$a_1<a_2<a_3$,一定存在唯一的$f$,它在上半平面$D$中单叶且全纯,在$\bar D$上连续,把$D$一一地映为多角形域$G$,且把$a_1,a_2,a_3$分别映为$G$的三个顶点$w_1,w_2,w_3$.
\end{theorem}
\begin{proof}
根据引理 \ref{lemma7.4.1},存在$w=\varphi(\zeta)$,它在$\zeta$平面的上半平面$\Im\zeta>0$中单叶且全纯,在$\Im\zeta\ge0$上连续,且把$\Im\zeta>0$一一地映为$G$,把$\Im\zeta=0$一一地映为$\partial G$.设$\varphi^{-1}(w_j)=\zeta_j,j=1,2,3$,且满足$\zeta_1<\zeta_2<\zeta_3$.作分式线性变换$\zeta=T(z)$,把$a_1,a_2,a_3$分别变成$\zeta_1,\zeta_2$和$\zeta_3$,于是复合函数
\[w=\varphi\big(T(z)\big)=f(z)\]
便满足定理的要求.

如果还有$w=g(z)$也满足定理的要求,那么函数$h(z)=g^{-1}\big(f(z)\big)$便把上半平面映为上半平面,且把实轴映为实轴.如果记$g(\infty)=w'$,则$w'$是多角形边界上某一点,而$f^{-1}(w')=t$,$t$是实轴上某一点,那么$h(t)=\infty$.现在$h$在$(-\infty,t)$和$(t,\infty)$上取实数值,根据对称原理,它可以全纯开拓到下半平面,因而$h\in\Aut(\MC_\infty)$.由定理 \ref{thm5.3.5},$h$是一个分式线性变换.但$h(z_j)=z_j,j=1,2,3$,即$h$有三个不动点,因而$h(z)=z$,即$f(z)=g(z)$.这就证明了满足定理要求的函数是唯一的.
\end{proof}

如果$a_1,\cdots,a_n$中有一个是$\infty$,比如$a_n=\infty$,这时Schwarz-
Christoffel公式是一个什么样子呢?我们有下面的
\begin{theorem}\label{thm7.4.5}
$f$如定理 \ref{thm7.4.3} 中所述,但与$w_n$对应的$a_n=\infty$,这时$f$可表示为
\begin{equation}\label{eq7.4.8}
f(z)=C\int_{z_0}^z(z-a_1)^{\alpha_1-1}\cdots(z-a_{n-1})^{\alpha_{n-1}-1}\dz+C_1.
\end{equation}
\end{theorem}
\begin{proof}
任取$a<a_1$,作分式线性变换
\[\zeta=T(z)=\frac1{a-z}.\]
容易看出,它把$\Im z>0$映为$\Im\zeta>0$,把实轴上的
\[a_1<a_2<\cdots<a_{n-1},a_n=\infty\]
映为
\[b_1<b_2<\cdots<b_{n-1}<b_n=0,\]
这里,$b_j=T(a_j),j=1,\cdots,n$. 于是,函数
\[F(\zeta)=(f\circ T^{-1})(\zeta)\]
把上半平面$\Im\zeta>0$双全纯地映为多角形域$G$,且把$b_1,\cdots,b_n$.分别映为$w_1,\cdots,w_n$. 根据定理 \ref{thm7.4.3},$F$可表示为
\[F(\zeta)=C'\int_{\zeta_0}^\zeta(\tau-b_1)^{\alpha_1-1}\cdots
(\tau-b_{n-1})^{\alpha_{n-1}-1}\tau^{\alpha_n-1}\textrm d\tau+C_1.\]
作变量代换$\tau=\frac1{a-\eta}$,因为$b_k=\frac1{a-a_k},k=1,\cdots,n-1$,所以
\[\tau-b_k=\frac{\eta-a_k}{(a-\eta)(a-a_k)},\;k=1,\cdots,n-1.\]
上式就变成
\[
F(\zeta)=C\int_{T^{-1}(\zeta_0)}^{T^{-1}(\zeta)}(\eta-a_1)^{\alpha_1-1}\cdots(\eta-a_{n-1})
^{\alpha_{n-1}-1}\cdot\frac1{(a-\eta)^{\alpha_1+\cdots+\alpha_n-(n-2)}}\textrm d\eta+C_1,
\]
这里,$C=\frac{C'}{(a-a_1)^{a_1-1}\cdots(a-a_{n-1})^{\alpha_{n-1}-1}}$. 由于$\alpha_1+\cdots+\alpha_n=n-2$,所以
\[F(\zeta)=C\int_{T^{-1}(\zeta_0)}^{T^{-1}(\zeta)}(\eta-a_1)^{\alpha_1-1}\cdots(\eta-a_{n-1})
^{\alpha_{n-1}-1}\textrm d\eta+C_1.\]
由此即得
\begin{align*}
f(z)&=F\big(T(z)\big)\\
&=C\int_{z_0}^z(\eta-a_1)^{\alpha_1-1}\cdots(\eta-a_{n-1})^{\alpha_{n-1}-1}\textrm d\eta+C_1.
\end{align*}
这就是要证明的.
\end{proof}

定理 \ref{thm7.4.5} 说明,如果$a_n=\infty$,那么在公式中把$(z-a_n)^{\alpha_n-1}$这个因子去掉就行了.

如果多角形有一个或几个顶点在无穷远处,变换的公式应是什么样子呢?
\begin{figure}[!ht]
\centering
\begin{tikzpicture}[thick,every node/.style={inner sep=2pt},scale=1.4,
>={Stealth[width=3pt]}]
\coordinate(O)at(0,0);\coordinate(A)at(30:2);
\coordinate(B)at(90:1.2);\coordinate(C)at(150:1.1);
\coordinate(D)at(210:1);\coordinate(E)at(210:2.4);
\coordinate(F)at(210:2.8);\coordinate(G)at(256:2.1);
\coordinate(H)at(256:2.4);\coordinate(I)at(256:1);
\coordinate(J)at(330:1);\coordinate(K)at(-30:2.1);
\draw(A)--(B)--(C)--(D)--(E)node[above left]{$w_k'$}--(G)node[ right]{$w_k''$}--(I)--(J)--(K)--cycle;
\draw[densely dashed](0,0)node[above right]{$\beta_k\pi$}
--(D)node[above left]{$w_{k-1}$}(O)--(I)node[below right]{$w_{k+1}$};
\draw(E)--(F)(G)--(H)(230:2.8)node{$w_k$};
\begin{scope}
\clip(D)--(E)--(G);
\draw(E)circle(0.25);
\node at(217:1.93){$\alpha_k'\pi$};
\end{scope}
\begin{scope}
\clip(E)--(G)--(I);
\draw(G)circle(0.25);
\node at(246:1.8){$\alpha_k''\pi$};
\end{scope}

\end{tikzpicture}
\caption{\label{fig7.7}}
\end{figure}

设$w_k=\infty$. 如图 \ref{fig7.7} 所示,在$w_{k-1}w_k$和$w_kw_{k+1}$上分别取$w_k'$和$w_k''$,连接$w_k'$和$w_k''$后得一$n+1$角形$G'$. 于是,把上半平面映为$G'$的公式为
\[w=C\int_{z_0}^z(z-a_1)^{\alpha_1-1}\cdots(z-a_k')^{\alpha_k'-1}(z-a_k'')^{\alpha_k''-1}
\cdots(z-a_n)^{\alpha_n-1}\dz+C_1,\]
这里,$a_k'$和$a_k''$是$w_k'$和$w_k''$在实轴上的原像. 现在让线段$w_k'w_k''$平行移动到无穷远去,$G'$就趋于$G$,而$a_k'$和$a_k''$就趋于$\infty$点在实轴上的原像$a_k$,上面公式中的被积函数
\[(z-a_k')^{\alpha_k'-1}(z-a_k'')^{\alpha_k''-1}\to(z-a_k)^{\alpha_k'+\alpha_k''-2}.\]
如果直线$w_{k-1}w_k'$和$w_k''w_{k+1}$在有限点处的交角为$\beta_k\pi$,那么易知$\alpha_k'+\alpha_k''+\beta_k=1$,所以$\alpha_k'+\alpha_k''-2=-1-\beta_k$. 若记$\alpha_k=-\beta_k$,则有
\[\alpha_k'+\alpha_k''-2=\alpha_k-1.\]
这样,上面的公式仍可写成
\[w=C\int_{z_0}^z(z-a_1)^{\alpha_1-1}\cdots(z-a_k)^{\alpha_k-1}\cdots(z-a_n)^{\alpha_n-1}
\dz+C_1.\]
这里的$\alpha_k\pi$是顶点在无穷远处的两条边在有限点处的交角乘以$-1$.综上所述,我们得到了
\begin{theorem}\label{thm7.4.6}
设$G$是有一个或几个顶点在无穷远处的多角形,如果把在$\infty$处那个顶点的顶角规定为过$\infty$点的两条边在有限点处的夹角乘以$-1$,那么Schwarz--Christoffel公式仍然成立.
\end{theorem}

下面通过例子来说明Schwarz--Christoffel公式的用法.
\begin{example}\label{thm7.4.7}
设$G$是以$w_1,w_2,w_3$为顶点,相应的顶角为$\alpha_1\pi,\alpha_2\pi,\alpha_3\pi$($\alpha_1+\alpha_2+\alpha_3=1$)的三角形域,求出把上半平面映为$G$的保角变换.
\end{example}
\begin{solution}
设所求的变换为$w=f(z)$. 在实轴上取三点$a_1=0,a_2=1,a_3=\infty$,使得$f(a_j)=w_j,j=1,2,3$. 由公式 \eqref{eq7.4.8},有
\begin{equation}\label{eq7.4.9}
f(z)=C\int_0^z\zeta^{\alpha_1-1}(1-\zeta)^{\alpha_2-1}\textrm d\zeta+C_1.
\end{equation}
由$f(0)=w_1$,得$C_1=w_1$. 再由$f(1)=w_2$,得
\[C=\frac{w_2-w_1}{\int_0^1t^{\alpha_1-1}(1-t)^{\alpha_2-1}\textrm dt}
=\frac{w_2-w_1}{B(\alpha_1,\alpha_2)}.\]
这里,$B$表示Beta函数. 由于$\alpha_1+\alpha_2+\alpha_3=1$,所以
\[B(\alpha_1,\alpha_2)=\frac{\Gamma(\alpha_1)\Gamma(\alpha_2)}{\Gamma(\alpha_1+\alpha_2)}
=\frac{\Gamma(\alpha_1)\Gamma(\alpha_2)}{\Gamma(1-\alpha_3)}.\]
由余元公式,有
\[\Gamma(\alpha_3)\Gamma(1-\alpha_3)=\frac\pi{\sin\alpha_3\pi},\]
所以
\[B(\alpha_1,\alpha_2)=\frac1\pi\Gamma(\alpha_1)\Gamma(\alpha_2)
\Gamma(\alpha_3)\sin\alpha_3\pi.\]
于是
\[C=\frac{\pi(w_2-w_1)}{\Gamma(\alpha_1)\Gamma(\alpha_2)
\Gamma(\alpha_3)\sin\alpha_3\pi}.\]
将$C$和$C_1$的值代入 \eqref{eq7.4.9} 式,即得所要求的变换.
\end{solution}

\begin{example}\label{exam7.4.8}
求一保角变换,把上半平面映为域$G=\bigg\{w:-\frac\pi2<\Re w<\frac\pi2,\Im w>0\bigg\}$(图 \ref{fig7.8}).
\end{example}
\noindent\begin{minipage}{0.6\textwidth}
\begin{solution}
把$G$看成三角形,它的三个顶点为
\[w_1=-\frac\pi2,w_2=\frac\pi2,w_3=\infty,\]
对应的三个角分别为$\frac\pi2,\frac\pi2$和$0$. 在实轴上取三个点$a_1=-1,a_2=1,a_3=\infty$,由公式 \eqref{eq7.4.8} 即知,把上半平面映为域$G$的变换为
\begin{align*}
f(z)&=C\int_0^z(z+1)^{-\frac12}(z-1)^{-\frac12}\dz+C_1\\
&=C\int_0^z\frac{\dz}{\sqrt{z^2-1}}+C_1=C'\arcsin z+C_1.
\end{align*}
由$f(-1)=-\frac\pi2,f(1)=\frac\pi2$,得$C_1=0,C'=1$. 故所求的变换为
\begin{equation*}
w=\arcsin z.
\end{equation*}
\end{solution}
\end{minipage}%
\begin{minipage}{0.4\textwidth}
\centering
\begin{tikzpicture}[thick,every node/.style={inner sep=2pt},scale=1.4,
>={Stealth[width=3pt]}]
\draw(-pi/2,pi)--(-pi/2,0)node[left]{$w_1$}node[below]{$-\frac\pi2$}
--(0,0)node[below]{$O$}--(pi/2,0)node[below]{$\frac\pi2$}node[right]{$w_2$}
--(pi/2,pi)(0,pi)node{$w_3$};
\fill(0,0)circle(0.7pt);
\end{tikzpicture}
\captionof{figure}{\label{fig7.8}}
\end{minipage}

\begin{example}\label{exam7.4.9}
求一保角变换,把带状域$D=\{z:-\pi<\Im z<\pi\}$映为图 \ref{fig7.9}\subref{fig7.9b} 所示的域$G$.
\end{example}
\begin{solution}
由对称性,我们只需考虑把$0<\Im z<\pi$映为图 \ref{fig7.9}\subref{fig7.9b} 中上半部分域的变换. 因为$\zeta=\ee^z$可以把$0<\Im z<\pi$映为上半平面,故只需研究把上半平面映为图 \ref{fig7.10}\subref{fig7.10b} 中的那个域$G'$.

\begin{figure}[!ht]
\centering
\subcaptionbox{\label{fig7.9a}}[0.48\textwidth]
{
\begin{tikzpicture}[thick,every node/.style={inner sep=2pt},
>={Stealth[width=3pt]}]
\draw[->](0,-2)--(0,-1)node[below right]{$\pi\textrm i$}--(0,0)node[below right]{$O$}--(0,1)node[above right]{$\pi\textrm i$}--(0,2);
\draw[densely dashed](-2.2,0)--(2.2,0);
\draw(-2.2,1)--(2.2,1)(-2.2,-1)--(2.2,-1);
\end{tikzpicture}
}
\subcaptionbox{\label{fig7.9b}}[0.48\textwidth]
{
\begin{tikzpicture}[thick,every node/.style={inner sep=2pt},
>={Stealth[width=3pt]}]
\draw[->](0,-2)--(0,-1)node[below left]{$-h\textrm i$}--(0,0)node[below right]{$O$}--(0,1)node[above left]{$h\textrm i$}--(0,2);
\draw(-2.2,0)--(2.2,0);
\draw(-2.2,1)--(0,1)--++(30:2)(-2.2,-1)--(0,-1)--++(-30:2);
\draw[densely dashed](0,1)--(2.2,1)(0,-1)--(2.2,-1);
\draw(0.4,1)arc(0:30:0.4)(0.4,-1)arc(0:-30:0.4);
\draw(0.9,1.22)node{$\beta\pi$}(0.9,-1.22)node{$\beta\pi$};
\end{tikzpicture}
}
\caption{\label{fig7.9}}
\end{figure}
\begin{figure}[!ht]
\centering
\subcaptionbox{\label{fig7.10a}}[0.48\textwidth]
{
\begin{tikzpicture}[thick,every node/.style={inner sep=2pt},scale=1.5,
>={Stealth[width=3pt]}]
\draw(-1.5,0)--(-1,0)node[below]{$-1$}--(-0.5,0)node[below]{$z_1$}
--(0,0)node[below]{$O$}--(0.5,0)node[below]{$z_2$}--(1.5,0);
\draw(0.5,0)arc(0:180:0.5);
\draw(0,0)--(45:0.5)node[above right]{$\rho$};
\fill(-1,0)circle(0.65pt)(0,0)circle(0.65pt);
\end{tikzpicture}
}
\subcaptionbox{\label{fig7.10b}}[0.48\textwidth]
{
\begin{tikzpicture}[thick,every node/.style={inner sep=2pt},
>={Stealth[width=3pt]}]
\draw(-3,0)node[above left]{$w_1$}--(0,0)node[below]{$O$}--(2,0)node[above right]{$w_2$};
\draw[->](0,0)--(0,1)node[above left]{$w_3=h\textrm i$}--(0,2.5);
\draw(-3,1)--(0,1)--++(35:2);
\draw[densely dashed](0,1)--(2,1);
\draw(0.4,1)arc(0:35:0.4)(0.7,1.2)node{$\beta\pi$};
\end{tikzpicture}
}
\caption{\label{fig7.10}}
\end{figure}

把$G'$看成一个三角形,它的三个顶点分别为
\[w_1=\infty,w_2=\infty,w_3=h\textrm i,\]
相应的三个角分别为$\alpha_1=0,\alpha_2=-\beta$和$\alpha_3=\beta+1$. 让$a_1=0,a_2=\infty,a_3=-1$,由定理 \ref{thm7.4.6},所求的变换为
\begin{equation}\label{eq7.4.10}
w=f(\zeta)=C\int_{-1}^\zeta z^{-1}(z+1)^\beta\textrm dz+C_1.
\end{equation}
因为$f(-1)=h\textrm i$,所以$C_1=h\textrm i$.

现设法确定常数$C$. 以$a_1=0$为圆心、$\rho$为半径作圆,与实轴交于$z_1$和$z_2$(图 \ref{fig7.10}\subref{fig7.10a}). 因为$z_1\in(-1,0)=(a_3,a_1),z_2\in(0,\infty)=(a_1,a_2)$,所以$f(z_1)\in w_3w_1,f(z_2)\in w_1w_2$,因而
\begin{equation}\label{eq7.4.11}
\Im\big(f(z_2)-f(z_1)\big)=-h.
\end{equation}
另一方面,如果记$\gamma_\rho=\{z:z=\rho\ee^{\ii\theta},0\le\theta\le\pi\}$,则当$z\in\gamma_\rho$时,有
\[(1+z)^\beta=(1+\rho\ee^{\ii\theta})^\beta=1+\beta\rho\ee^{\ii\theta}+\cdots
=1+O(\rho).\]
于是,从 \eqref{eq7.4.10} 式可得
\begin{align*}
f(z_2)-f(z_1)&=C\int_{z_1}^{z_2}\frac{(1+z)^\beta}z\dz=-C\int\limits_{\gamma_\rho}
\frac{(1+z)^\beta}z\dz\\
&=-C\int_0^\pi\frac{(1+\rho\ee^{\ii\theta})^\beta}{\rho\ee^{\ii\theta}}
\rho\ii\ee^{\ii\theta}\textrm d\theta\\
&=-C\pi\ii+O(\rho),
\end{align*}
令$\rho\to0$,即得
\begin{equation}\label{eq7.4.12}
\Im\big(f(z_2)-f(z_1)\big)=-C\pi.
\end{equation}
比较 \eqref{eq7.4.11} 式和 \eqref{eq7.4.12} 式,即得
\[C=\frac h\pi.\]

至此,得到把上半平面$\Im\zeta>0$映为$G'$的变换为
\begin{equation}\label{eq7.4.13}
w=f(\zeta)=\frac h\pi\int_{-1}^\zeta\frac{(1+z)^\beta}z\dz+\ii h.
\end{equation}
所以,$w=f(\ee^z)$是把$0<\Im z<\pi$映为$G'$的变换. 当$z$在实轴上时,$\ee^z$在正实轴上,因而$f(\ee^z)$取实数值,根据对称原理,$w=f(\ee^z)$把$-\pi<\Im z<\pi$映为$G$.

下面来看两个特例:

(1) 如果$\beta=1$,那么
\[f(\zeta)=\frac h\pi\int_{-1}^\zeta\bigg(1+\frac1z\bigg)\dz+\ii h
=\frac h\pi(\log\zeta+\zeta+1),\]
所以
\[w=f(\ee^z)=\frac h\pi(\ee^z+z+1),\]
它把$-\pi<\Im z<\pi$一一地映为全平面除去两条半直线(图 \ref{fig7.11}):$\{z:z=x\pm h\ii,-\infty<x<0\}$.
\begin{figure}[!ht]
\centering
\begin{minipage}[b]{0.48\linewidth}
\centering
\begin{tikzpicture}[thick,every node/.style={inner sep=2pt},
>={Stealth[width=3pt]}]
\draw(-2,0.7)--(2,0.7)node[above]{$h\textrm i$};
\draw(-2,-0.7)--(2,-0.7)node[below]{$-h\textrm i$};
\draw[densely dashed](-2,0)--(2,0)node[right]{$O$};
\fill(2,0)circle(1pt)(2,-0.7)circle(1pt)(2,0.7)circle(1pt);
\end{tikzpicture}
\caption{\label{fig7.11}}
\end{minipage}\hfill
\begin{minipage}[b]{0.48\linewidth}
\centering
\begin{tikzpicture}[thick,every node/.style={inner sep=2pt},
>={Stealth[width=3pt]}]
\draw(-2,0.7)--(2,0.7)node[right]{$h\textrm i$}--(2,2.2);
\draw(-2,-0.7)--(2,-0.7)node[right]{$-h\textrm i$}--(2,-2.2);
\draw[densely dashed](-2,0)--(2,0)node[right]{$O$};
\fill(2,0)circle(1pt)(2,-0.7)circle(1pt)(2,0.7)circle(1pt);
\end{tikzpicture}
\caption{\label{fig7.12}}
\end{minipage}
\end{figure}

(2) 如果$\beta=\frac12$,那么
\begin{align*}
f(\zeta)&=\frac h\pi\int_{-1}^\zeta\frac{\sqrt{z+1}}z\dz+\ii h\\
&=\frac{2h}\pi\bigg[\sqrt{\zeta+1}+\log\big(\sqrt{\zeta+1}-1\big)-\frac12
\log\zeta\bigg],
\end{align*}
所以
\[w=f(\ee^z)=\frac{2h}\pi\bigg[\sqrt{\ee^z+1}+\log\big(\sqrt{\ee^z+1}-1\big)
-\frac z2\bigg],\]
它把$-\pi<\Im z<\pi$一一地映为图 \ref{fig7.12} 所示的域.
\end{solution}

\begin{xiti}
\item 设$L_k=\big\{z=r\ee^{\ii\frac{2k\pi}n}:1\le r<\infty\big\},G=\MC\backslash
\big(\bigcup_{k=1}^nL_k\big)$.求出将上半平面映为$G$的双全纯映射$\varphi$,使得$\varphi(\ii)=0$.
\[\Bigg(\varphi(z)=\frac{z^2+1}{\bigg(\sum_{k=0}^{[\frac n2]}
(-1)^k\binom n{2k}z^{n-2k}\bigg)^{\frac2n}}\Bigg)\]
\item 证明:将$B(0,1)$映为$n$角形内部的双全纯映射$\varphi$具有形状
\[\varphi(z)=C_0\int_{z_0}^z\prod_{k=1}^n(\zeta-a_k)^{\alpha_k-1}\textrm d\zeta+C_1,\]
其中,$a_1,a_2,\cdots,a_n\in\partial B(0,1)$是与该$n$角形顶点相对应的点;$\alpha_1\pi,\alpha_2\pi,\cdots,\alpha_n\pi$是该$n$角形的内角:$z_0\in B(0,1)$.
\item 设$G$是正$n$角形,$O$是它的中心,$1$是它的一个顶点.求出将$B(0,1)$映为$G$的双全纯映射$\varphi$,使得$\varphi(0)=0,\varphi'(0)>0$.
\[\Bigg(\varphi(z)=\frac n{B(\frac1n,1-\frac2n)}\int_0^z(1-\zeta^n)^{-\frac2n}\textrm d\zeta\Bigg)\]
\item 证明:将上半平面映为$\MC_\infty$中的$n$角形外部的单叶亚纯函数$\varphi$具有形状
\[\varphi(z)=C_0\int_{z_0}^z\prod_{k=1}^n(\zeta-a_k)^{\alpha_k-1}
\frac1{(\zeta-a_0)(\zeta-\bar{a_0})}\textrm d\zeta+C_1,\]
其中,$a_1,a_2,\cdots,a_n\in\MR$是与该$n$角形顶点相对应的点;$\alpha_1\pi,\alpha_2\pi,\cdots,\alpha_n\pi$是该$n$角形的外角;$a_0=\varphi^{-1}(\infty),\Im z_0>0$.
\item 设$G=\MC_\infty\backslash\big([0,1]\cup[0,\ee^{\ii\frac{2\pi}3}]\cup
[0,\ee^{-\ii\frac{2\pi}3}]\big),\Im z_0>0$.求出将上半平面映为$G$的单叶亚纯映射$\varphi$,使得$\varphi(z_0)=\infty$.
\[\bigg(\varphi(z)=\frac{[(z-z_0)^3-3y_0^2(z-x_0)]^{\frac23}}{(z-x_0)^2+y_0^2}\bigg)\]
\item 设$L_1,L_2,\cdots,L_n$是以$O$为端点的$n$条线段,它们彼此仅在$O$处相交,$\Im a_0>0$.证明:将上半平面映为$G=\MC_\infty\backslash\big(\bigcup_{k=1}^n
    L_k\big)$,并且将$a_0$映为$\infty$的单叶亚纯函数$\varphi$具有形状
\[\varphi(z)=C\frac{\prod_{k=1}^n(z-a_k)^{\alpha_k}}{(z-a_0)(z-\bar{a_0})},\]
其中,$a_1,a_2,\cdots,a_n\in\MR;\alpha_1,\alpha_2,\cdots,\alpha_n>0$,并且$\sum_{k=1}^n\alpha_k=2$.

\end{xiti}
